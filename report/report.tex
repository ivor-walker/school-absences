% 1 Overview
% This piece of coursework involves gaining knowledge and experience by working with Apache Spark in order to analyse a large dataset. 

% 2 Practical Requirements
% You will develop a console-based application in Python and Apache Spark in order to analyse a large dataset. The dataset (published by the UK government) describes pupil absence in schools in England from 2006-2018 [1]. The dataset contains over 280,000 records with multiple data recorded, such as the number of schools, number of pupils, the region/local authority, school type, number of authorised absences. You should perform an exploration of the dataset beforehand, making sure you fully understand the structure. Also, you will have to think about ways in which you will deal with the various columns, missing/unnecessary data and anything else you may identify in the task.  

% 2.1 Part 1

% • Read the dataset using Apache Spark.
% • Store the dataset using the methods supported by Apache Spark.

% • Allow the user to search the dataset by the local authority, showing the number of pupil enrolments in each local authority by time period (year).
% 	– Given a list of local authorities, display in a well-formatted fashion the number of pupil enrolments in each local authority by time period (year).

% • Allow the user to search the dataset by school type, showing the total number of pupils who were given authorised absences in a specific time period (year).
% 	– Extend this by allowing the user to further see the breakdown of specific types of authorised absences given to pupils by school type in a specific time period (year).

% • Allow a user to search for all unauthorised absences in a certain year, broken down by either region name or local authority name.


% 2.2 Part 2



% 2.3 Part 3


% 4 Deliverables

% • A short report (no more than 1,500 words with a 1-mark penalty applied per every 10% over the limit) discussing your end-to-end solution, your approach/implementation and the answers to the questions posed + appropriate analysis. Your report should also include the specific code lines that match the functionality you implemented as well as:
% – A table containing the features you implemented.
% – Any problems that you encountered and your solutions.
% – Any diagram you feel is necessary.

\documentclass{article}
\usepackage[
    left=2.5cm,
    right=2.5cm,
    top=2.5cm,
    bottom=2.5cm,
]{geometry}

\usepackage{hyperref}
\usepackage{graphicx}

\usepackage{tabularx}
\usepackage{longtable}
\usepackage{ltablex}
\keepXColumns

\usepackage{amsmath}

% Disable paragraph indentation
\setlength{\parindent}{0pt}

\title{School's Out: A Console-Based Application for Analysing Pupil Absence Data}
\author{Ivor Walker}
\date{}

\begin{document}

\maketitle
Word Count: 1,437

\newpage

\section*{Introduction}
Understanding drivers behind pupil attendance enables governments to reward the highest performing schools, support those that are struggling, and ultimately prioritise resources towards initiatives that affect the most change in pupil attendance. This report details the development of a console-based application in Python and Apache Spark to analyse a large dataset of pupil absence in schools in England from 2006-2018. The application meets all requirements listed in Part 1, 2, and 3. \ref{tab:requirements}

\section*{Design}
    \subsection*{main.py and Menu}
    main.py is the entry point of the application, which creates an instance of the Menu class. The Menu class starts the main menu loop, asking users for which analysis they would like to perform, getting the required data from the Absence class, and displaying the results using the View class. All expected exceptions (e.g View raising an invalid user input or SparkData raising a nonexistent local authority) are caught here.

    \subsection*{View}
        The View class contains methods to display tables returned from the Absences class in a readable table format or chart, and methods to request input from the user. When the Menu class asks for user input, it also passes asses the expected format of the user input to the View class, which validates the input and returns it to the Menu class.

    \subsection*{SparkData}

        \subsubsection*{Constructor}
        The constructor of SparkData starts the Spark session and loads a csv at the provided location into a Spark Dataframe. The user can input strings (e.g for analyses for a specific local authority) which should be case insensitive, but the data itself is in a specific case format. After loading data, I infer the case of each column based on a sample of each column (n = 50 by default) and create a map of all string columns with their case. I tranform any user input to the correct case of the requested column before searching. Certain supposedly numeric values (e.g "sess\_auth\_ext\_holiday") sometimes contained symbols, which I infer to be missing data and replace with 0. I round all numeric values to five decimal places to improve output readability.
        \subsubsection*{Getting subframes}
        SparkData contains methods to return subsets of any DataFrame in certain formats and aggregations. 
        \_get\_frame returns all requested columns of the DataFrame, filtered by where the supplied filter\_cols is in the corresponding list of values in filter\_passes. 
    \_get\_agg\_frame returns a \_\_get\_grouped\_frame of a \_get\_frame subframe. \_\_get\_grouped\_frame is a transformation of any DataFrame that pivots on one specific column and aggregated (counts, sums, or mean) by another column. \_get\_multi\_col\_agg\_frame stacks the columns of a \_get\_frame subframe, unpivots it, and returns a \_\_get\_grouped\_frame of the unpivoted DataFrame - effectively aggregating by all columns selected in the original \_get\_frame query. 
        Certain functionality cannot use Spark Dataframes as input (e.g matplotlib for charting), so \_collect\_to\_dict takes an input DataFrame, collects it to the driver, computes means and confidence intervals, and returns a usable dictionary of the results. \\

    \subsection*{Absences}
    Absences inherits from SparkData and passes the location of the absences file to SparkData's construction. It contains methods called by Menu to return data required specifically for analyses of absence data by wrapping around the subframe methods in SparkData. 

    It sets default values of filters which are always passed to \_get\_frame, e.g always excluding school type and geographical aggregation rows in the returned frames, and fetches some values for specific columns in the dataset which can be used instead of user input to simplify testing.

    \subsection*{Comparing local authorities}
    % • Allow a user to compare two local authorities of their choosing in a given year. Justify how you will compare and present the data.
    Comparing local authorities by overall absence rate adjusts for the size of the local authority which could otherwise skew our results. The goal of governments is to reduce overall absence rates, but authorised absence rates are less of a problem because they represent unavoidable absences whereas unauthorised absences represent some disconnect between the school and the pupil. I compare all three figures to assess the performance of local authorities.

    Students commit absences at different rates, and modelling showed that the percentage of enrolments that are persistent absentees has the largest effect on overall absence rates outside of time period and region. \ref{tab:model_output} I show this figure to partially explain the overall absence rate, and compare percentages of unauthorised absences by persistent absentees to show the proportion of pupils that are totally disconnected from the school and need the most support.

    If the user chooses to compare more than five local authorities, I add a confidence interval around the mean of each category to enable conclusions on statistical significance, e.g a council has a higher percentage of unauthorised absences than the average council chosen.

    \subsection*{Methodology for modelling absence rates}
    % -Model didn't fit. Logged the offset & added "dropLast" to dummy variables
    % -Model fitted with warnings of singular covariance matrix . Added checks for perfect multicollinearity.
    % -all_through was useless (always "All through"). Dropped all_through
    % -Levels of school_type and academy_type were complementary. Removed all academy related covariates
    
    % Final model: absence_sessions ~ Poisson(lambda)
    % log(lambda) / log(possible_sessions) = BX
    % B = [time_period, region_name, school_type, enrolments, unauthorised_percent, proportion of absences that are by persistent absentees, proportion of unauthorized absences that are by persistent absentees, percentage of enrolments that are persistent absentees]
    Alongside visualisations, I fit a model to precisely quantify how changes in regions and school types affect absences, and capture variance from other variables. I also included interaction terms between region and school type to explroe the effect of the relationship between region and school type on absences.
    
    Absences contains methods for producing data required for modelling, a modelling framework, and a fitted model explaining overall absences. A Generalised Linear Model (GLM) retains explainable coefficients which is useful for explaining our response. A Poisson distribution fits our response variable best because it is a count, but I applied an offset of the number of possible sessions to account for different school sizes so we are effectively modelling the absence rate.
    
    I tried to use all covariates initially, but the model failed to fit because some variables were perfectly "multicolinear" (i.e a covariate was related to another covariate in the data). I created a "\_print\_high\_collinearity" method to identify these variables and found that any school that had an associated academy type was a state-funded school. I removed all academy related covariates and the model fitted successfully. My final model is:
    \begin{equation}
    \begin{aligned}
\text{overall absence sessions} &\sim \text{Poisson}(\lambda) \\
\log(\lambda) - \log(\text{total possible sessions}) &= \eta \\
        \eta &= \beta_0 + \beta_1 \text{time period} + \beta_2 \text{region} &+ \beta_3 \text{school type} \\
        + \beta_4 \text{enrolments} &+ \beta_5 \text{\% unauthorised sessions} \\
        + \beta_6 \text{\% overall absences committed by persistent absentees} &+ \beta_7 \text{\% unauthorized absences committed by persistent absentees} \\
        &+ \beta_8 \text{\% enrolments that are persistent absentees} \\
    \end{aligned}
    \end{equation}

\section*{Results}
    \subsection*{Analysing pupil attendance by region over time}
    % • Chart/explore the performance of regions in England from 2006-2018. Your charts and subsequent analysis in your report should answer the following questions:
    % 	– Are there any regions that have improved in pupil attendance over the years?
    % 	– Are there any regions that have worsened?
    % 	– Which is the overall best/worst region for pupil attendance?
        \subsubsection*{Attendance for all regions over time} 
        I chose overall absence rate as the metric to represent pupil attendance to account for the different sizes of regions. Similarly, I compared the absence rate for each region against the mean absence rate for all regions instead of against the national absence rate to avoid larger regions skewing the results.

        All regions has a significantly lower absence rate in 2018/19 than the mean council in 2006/07. The absence rate for the mean council falls drastically between 2006/07 and 2013/14, almost always being significantly lower than the previous year, but increases from 2014/15 until 2018/19. This increase is more intermittent and smaller in scale, with the new mean level remaining significantly lower than any year before 2013/14. \ref{fig:absence_rate_over_time_by_region}

        \subsubsection*{Improving and declining regions}
        Outer and Inner London has absence rates from 2018/19 that are within the confidence intervals for 2013/14, indicating these regions have significantly avoided the increased absence rates since 2014/15. Conversly, North East and Yorkshire and the Humber have absence rates from 2018/19 that are within the confidence intervals for 2011/12 and 2012/13, indicating these regions have not significantly benefitted from the overall decline in absence rates until 2013/14. \ref{fig:absence_rate_over_time_by_region}

        Overall, Outer London has the lowest absence rate across all years, whereas North East has the highest. \ref{fig:absence_rate_by_region}

    \subsection*{Exploring the relationship between school type, pupil absences and location}
    % • Explore whether there is a link between school type, pupil absences and the location of the school. For example, is it more likely that schools of type X will have more pupil absences in location Y? Write the code that performs this analysis, and write a paragraph in your report (with appropriate visualisations/charts) that highlight + explain your findings. 
        \subsubsection*{Overall absence rates by school type and region}
        Across all years, Inner London and Outer London have significantly lower absence rates than the mean region, whereas North East and Yorkshire and the Humber have significantly higher absence rates. \ref{fig:absence_rate_by_region} Special schools appear to have significantly higher absence rates than the mean school type \ref{fig:absence_rate_by_school_type}.
    
        \subsubsection*{Relationship between school type and region} 
    outh West and East Midlands have significantly lower state funded secondary schools and special schools, and higher primary schools than the mean region. Outer London and Inner London have significantly higher state-funded secondary schools and special schools, but significantly fewer state-funded primary schools. East of England, South West, East Midlands and Yorkshire and the Humber have significantly lower special schools. \ref{proportion_school_types_by_region}
    
    % {'feature': 'region_name_vec_Yorkshire and the Humber', 'coefficient': 1.022578074431429, 'lower': 1.020857952812246, 'upper': 1.0242981960506121}
    % {'feature': 'region_name_vec_West Midlands', 'coefficient': 1.022453969171505, 'lower': 1.020920140029721, 'upper': 1.0239877983132892}
    % {'feature': 'region_name_vec_South West', 'coefficient': 0.9835714170865579, 'lower': 0.9819054533598164, 'upper': 0.9852373808132995}
    % {'feature': 'region_name_vec_South East', 'coefficient': 1.010718470693176, 'lower': 1.0092498050974061, 'upper': 1.0121871362889459}
    % {'feature': 'region_name_vec_Outer London', 'coefficient': 0.975638543433034, 'lower': 0.9740346820153162, 'upper': 0.9772424048507519}
    % {'feature': 'region_name_vec_North West', 'coefficient': 0.9846662759973155, 'lower': 0.9831990712545243, 'upper': 0.9861334807401066}
    % {'feature': 'region_name_vec_North East', 'coefficient': 1.0092869795960528, 'lower': 1.0074748261785482, 'upper': 1.0110991330135575}
    % {'feature': 'region_name_vec_Inner London', 'coefficient': 0.9562152693031474, 'lower': 0.954397534764213, 'upper': 0.9580330038420818}
    % {'feature': 'region_name_vec_East of England', 'coefficient': 1.0254893121012958, 'lower': 1.0238311479641935, 'upper': 1.027147476238398}
    
    % {'feature': 'school_type_vec_State-funded secondary', 'coefficient': 0.9815238995422373, 'lower': 0.9802831270349176, 'upper': 0.982764672049557}
    % {'feature': 'school_type_vec_State-funded primary', 'coefficient': 0.9414428929574084, 'lower': 0.9402541524476719, 'upper': 0.9426316334671448}
    %
        \subsubsection*{Model results}
        All variables appear significant (see Discussion) but some have larger effect sizes than others. State funded secondary schools have a 2\% lower absence rate and state funded primary schools have a 6\% lower absence rate compared to special schools. Compared to East Midlands, some regions have higher absence rates (East of England: 3\%, Yorkshire and the Humber: 2\%, West Midlands: 2\%,  North West: 2\%, North East: 1\%, South East: 1\%) and others have lower absence rates (Inner London: 4\%, South West: 2\%, Outer London: 2\%). \ref{tab:model_output} These results differ from the previous analysis, e.g South West has an equal effect size as Outer London but did not have significantly lower absence rates than the mean region. This model is a more complete picture as it accounts for the effect of other variables on absence rates and the interaction components accounts for the relationship between school type and region. 

\section*{Discussion}
    \subsection*{Case sensitivity}
    The standard approach to guaranteeing case sensitivity is to set the user input to lowercase and search on a lowercase version of the data. The size of the dataset means producing lowercase versions of the data is not feasible. 
    
    I transform the user input to the correct case before searching to bypass this issue, but this approach is sensitive to the case being applied consistently throughout the column being searched, e.g. local authority names are inferred to be proper case and nearly all have a lowercase "of" (e.g City of London), but in the dataset 'Isles of Scilly' has a title case "Of" so searching for "Isles Of Scilly" would not return any results as it would be transformed to "Isles of Scilly".
    
    These issues are extremely rare and does not remove these mixed case names from aggregates which makes this tradeoff worth it. Searching for all violations of the case mapping and adding them to some "edge case" mapping would incur the same cost as producing a lowercase version of the data. As a compromise, I manually include encountered edge cases in Absences and inform SparkData to transform these edge cases to the exception before searching. This approach is appropriate here because of the vanishingly small number of edge cases, but manually listing edge cases is not scalable for datasets with larger number of edge cases.
    
    \subsection*{School types and absence rates}
    The chart proporting to show significant differences in absence rates between school types \ref{fig:absence_rate_by_school_type} includes the overall rate of absence, which biases the mean and associated confidence intervals downwards and creates the illusion that special schools have significantly higher absence rates than the mean school type. Comparing the differences between only three school types produces very large confidence intervals, so I included the total as a reference point. The proper statistical tool to compare the absence rates between school types is a t-test, but this is not implemented in Spark.

    \subsection*{Significance in the model}
% -Model fitted successfully but massive sample size made everything significant. No power analysis available. Use coefficients as effect size. 
    % -Model is overdispersed, spark doesn't support quasi-poisson/negative binomials
    The fitted model is overdispersed (estimated dispersion parameter: 152017105 / 276816 = 549.2, when it should be 1). Spark does not support quasi-poisson or negative binomial models that are needed to address this, so fixing this would require an external data analysis package which is forbidden by the spec. Consequently, the reported standard errors are underestimated leading to misleadingly small p-values. The model is also overpowered - the extremely large sample size is much higher than the minimum required to detect significance. Spark does not support power analysis required to find the minimum number of samples needed to detect a significant effect, but if I could perform this analysis I could drastically reduce the amount of data being handled to the minimum required. Because all covariates appear significant at this sample size and with this extreme overdispersion \ref{tab:model_output}, I focus on interpreting the coefficients as effect sizes which remain unbiased in the face of dispersion.

\section*{Reflective summary}
% – An additional reflective summary (not included in the word count) discussing the lessons learnt and experience that you have gained after finishing this practical.
    \subsection*{Data extraction and transformation}
    Although the SparkData class contains fewer methods and interactions with other classes, which are the typical sources of complexity, I spent the most time developing it because of the sheer level of 'precision' required when transforming data. Putting these transformation functions together to produce the desired output required relatively few lines of code, but how these transformation functions (e.g pivot) affect data was unintuitive because of how specific the transformations are to the data being transformed, and I spent a lot of time outside of the development environment thinking about how to transform the data to produce the desired output. However, I wish I spent more time thinking about exactly what transformations were required as I wasted time developing some methods (e.g the batch methods returning multiple DataFrames) that I later realised were not required.

    \subsection*{Program design}
    Creating generic methods to return the types of subframes required proved to be a good decision as it abstracted the somewhat confusing transformations away from myself and instead put data extraction in terms of the "rows" and "columns" of the DataFrames that I needed which simplified the framework sitting ontop of SparkData.

    \subsection*{Statistical analysis}
    I found the statistical tools that Spark offered to be frustratingly limited. Because the use of data analysis packages was forbidden in the assignment, I had to rely on simpler statistical tools e.g confidence intervals, and present flaws in my modelling approaches that I could not address because the appropriate procedure was not in Spark. Spark does not provide model diagnostics, so I had to implement methods to check for multicolinearity and overdispersion and could not check for other issues e.g outliers.

    The lack of interpretable non-linear modelling frameworks (e.g generalised additive models) restricted the temporal and spacial modelling that I could perform. For example, the same page that contains this dataset also contains a dataset listing school postcodes. If I had more sophisticated statistical tools, I could join this data onto the existing data, use an API (e.g postcodes.io) to convert these postcodes to latitude and longitude, and use this to more completely model the effect of location on absence rates instead of in a non-contiguous way as I have done here.

    Given that Spark is designed for large datasets, I found the lack of power analysis particularly strange as using the results from a power analysis could drastically reduce the amount of data that needs to be processed to discover relationships between data.

\newpage

\section*{Appendix}

\begin{table}[htbp]
    \centering
    \begin{tabularx}{\textwidth}{|X|X|X|X|X|X|}
        \hline
        \textbf{Part and bulletpoint in spec} & \textbf{Implemented feature to achieve spec} & \textbf{Line in Menu class} & \textbf{Line in Absences class} & \textbf{Line in SparkData class} & \textbf{Line in View class} \\  

        \hline
        \hline
        1a, 1b & Read and store dataset & 16 & 30 & 30 & NA \\
        \hline
        1c & Pupil enrolments over time by local authority & 88 & 146 & 542 &  134 \\
        \hline
        1d & Authorised absences by school type in year & 109 & 209 & 542 & 134 \\
        \hline
        1d extension & Authorised absence reasons by school type in year & 141 & 253 & 784 & 134 \\
        \hline
        1e & Unauthorised absences by region name or local authority in year & 172 & 359 & 542 & 134 \\

        \hline
        \hline
        2a & Compare local authorities in year & 209 & 422 & 784 & 134, 155 \\
        \hline
        2b & Absences by local authority over time & 247 & 477 & 542 & 134, 260 \\

        \hline
        \hline
        3 & Chart link between school type, pupil absences, and school location & 262 & 521, 558, 592 & 542 & 134, 260, 155 \\
        \hline
        3 & Model pupil absences using school type and school location & 262, 289 & 651, 636, 810, 793, 816 & 585 & 26 \\
        \hline
        \hline

    \end{tabularx}

    \caption{Table of all implemented requirements and their locations in the code. For example, the third bulletpoint of the first part of the spec asks to select pupil enrolments over time by local authority. Choosing this option in the menu runs line 88 in the Menu class, which gets data by calling line 146 in the Absences class (in turn calling line 542 in the SparkData class) and displays it by calling line 134 in the View class.}
    \label{tab:requirements}

\end{table}

% Coefficients:
    % Feature               Estimate   Std Error    T Value P Value
    % (Intercept)           -3.5445    0.0007 -5181.3648  0.0000
    % time_period_vec_2...  -0.0508    0.0001  -358.2155  0.0000
    % time_period_vec_2...  -0.0323    0.0001  -230.1206  0.0000
    % time_period_vec_2...  -0.0494    0.0001  -349.8540  0.0000
    % time_period_vec_2...  -0.0525    0.0001  -370.3205  0.0000
    % time_period_vec_2...  -0.0510    0.0001  -363.2750  0.0000
    % time_period_vec_2...  -0.0683    0.0001  -480.3252  0.0000
    % time_period_vec_2...   0.0019    0.0001    14.2779  0.0000
    % time_period_vec_2...  -0.0333    0.0001  -239.8553  0.0000
    % time_period_vec_2...   0.0065    0.0001    48.9744  0.0000
    % time_period_vec_2...   0.0124    0.0001    94.8431  0.0000
    % time_period_vec_2...   0.0087    0.0001    67.5511  0.0000
    % time_period_vec_2...   0.0057    0.0001    44.6063  0.0000
    % region_name_vec_Y...   0.0223    0.0009    26.0150  0.0000
    % region_name_vec_W...   0.0222    0.0008    29.0125  0.0000
    % region_name_vec_S...  -0.0166    0.0009   -19.1685  0.0000
    % region_name_vec_S...   0.0107    0.0007    14.3807  0.0000
    % region_name_vec_O...  -0.0247    0.0008   -29.4053  0.0000
    % region_name_vec_N...  -0.0155    0.0008   -20.3261  0.0000
    % region_name_vec_N...   0.0092    0.0009    10.0912  0.0000
    % region_name_vec_I...  -0.0448    0.0010   -46.1626  0.0000
    % region_name_vec_E...   0.0252    0.0008    30.5099  0.0000
    % school_type_vec_S...  -0.0186    0.0006   -28.9147  0.0000
    % school_type_vec_S...  -0.0603    0.0006   -93.6655  0.0000
    % enrolments             0.0000    0.0000   168.1255  0.0000
    % sess_unauthorised...  -0.0136    0.0000  -330.9179  0.0000
    % sess_overall_perc...   0.0108    0.0000  1037.0827  0.0000
    % sess_unauthorised...   0.0062    0.0000   444.7299  0.0000
    % enrolments_pa_10_...   0.0302    0.0000  5632.3827  0.0000
    % region_name_schoo...  -0.0194    0.0009   -22.2457  0.0000
    % region_name_schoo...  -0.0149    0.0009   -16.9720  0.0000
    % region_name_schoo...  -0.0252    0.0008   -32.2363  0.0000
    % region_name_schoo...  -0.0153    0.0008   -19.4537  0.0000
    % region_name_schoo...   0.0331    0.0009    37.5995  0.0000
    % region_name_schoo...   0.0499    0.0009    56.5946  0.0000
    % region_name_schoo...  -0.0027    0.0008    -3.6109  0.0003
    % region_name_schoo...  -0.0068    0.0008    -8.9870  0.0000
    % region_name_schoo...   0.0167    0.0009    19.5034  0.0000
    % region_name_schoo...   0.0277    0.0009    32.2775  0.0000
    % region_name_schoo...   0.0130    0.0008    16.7350  0.0000
    % region_name_schoo...   0.0129    0.0008    16.5428  0.0000
    % region_name_schoo...  -0.0068    0.0009    -7.2464  0.0000
    % region_name_schoo...   0.0138    0.0009    14.6669  0.0000
    % region_name_schoo...   0.0235    0.0010    23.6180  0.0000
    % region_name_schoo...   0.0415    0.0010    41.8981  0.0000
    % region_name_schoo...  -0.0166    0.0008   -19.6846  0.0000
    % region_name_schoo...  -0.0096    0.0008   -11.4366  0.0000
    % 
    % (Dispersion parameter for poisson family taken to be 1.0000)
    % Null deviance: 152017105.0272 on 276816 degrees of freedom
    % Residual deviance: 9875452.6146 on 276816 degrees of freedom
    % AIC: 12629286.0182
    % {'feature': 'time_period_vec_201819', 'coefficient': 0.9504595164989589, 'lower': 0.9501952804370483, 'upper': 0.9507237525608695}
    % {'feature': 'time_period_vec_201718', 'coefficient': 0.9682170911650351, 'lower': 0.9679507356633298, 'upper': 0.9684834466667404}
    % {'feature': 'time_period_vec_201617', 'coefficient': 0.9517574051582226, 'lower': 0.9514937606381002, 'upper': 0.952021049678345}
    % {'feature': 'time_period_vec_201516', 'coefficient': 0.9488874564342946, 'lower': 0.9486239669115923, 'upper': 0.9491509459569969}
    % {'feature': 'time_period_vec_201415', 'coefficient': 0.9502790702423299, 'lower': 0.9500175903700636, 'upper': 0.9505405501145963}
    % {'feature': 'time_period_vec_201314', 'coefficient': 0.9339581587122693, 'lower': 0.9336977718249025, 'upper': 0.934218545599636}
    % {'feature': 'time_period_vec_201213', 'coefficient': 1.001898333913589, 'lower': 1.0016374935980756, 'upper': 1.0021591742291023}
    % {'feature': 'time_period_vec_201112', 'coefficient': 0.9672716129855992, 'lower': 0.9670085949967379, 'upper': 0.9675346309744605}
    % {'feature': 'time_period_vec_201011', 'coefficient': 1.0065307710556763, 'lower': 1.0062685518628742, 'upper': 1.0067929902484785}
    % {'feature': 'time_period_vec_200910', 'coefficient': 1.0124412677103038, 'lower': 1.0121825671718634, 'upper': 1.0126999682487443}
    % {'feature': 'time_period_vec_200809', 'coefficient': 1.0087248817485996, 'lower': 1.0084706273923822, 'upper': 1.008979136104817}
    % {'feature': 'time_period_vec_200708', 'coefficient': 1.0057212294465723, 'lower': 1.0054691208707731, 'upper': 1.0059733380223714}
    % {'feature': 'region_name_vec_Yorkshire and the Humber', 'coefficient': 1.022578074431429, 'lower': 1.020857952812246, 'upper': 1.0242981960506121}
    % {'feature': 'region_name_vec_West Midlands', 'coefficient': 1.022453969171505, 'lower': 1.020920140029721, 'upper': 1.0239877983132892}
    % {'feature': 'region_name_vec_South West', 'coefficient': 0.9835714170865579, 'lower': 0.9819054533598164, 'upper': 0.9852373808132995}
    % {'feature': 'region_name_vec_South East', 'coefficient': 1.010718470693176, 'lower': 1.0092498050974061, 'upper': 1.0121871362889459}
    % {'feature': 'region_name_vec_Outer London', 'coefficient': 0.975638543433034, 'lower': 0.9740346820153162, 'upper': 0.9772424048507519}
    % {'feature': 'region_name_vec_North West', 'coefficient': 0.9846662759973155, 'lower': 0.9831990712545243, 'upper': 0.9861334807401066}
    % {'feature': 'region_name_vec_North East', 'coefficient': 1.0092869795960528, 'lower': 1.0074748261785482, 'upper': 1.0110991330135575}
    % {'feature': 'region_name_vec_Inner London', 'coefficient': 0.9562152693031474, 'lower': 0.954397534764213, 'upper': 0.9580330038420818}
    % {'feature': 'region_name_vec_East of England', 'coefficient': 1.0254893121012958, 'lower': 1.0238311479641935, 'upper': 1.027147476238398}
    % {'feature': 'school_type_vec_State-funded secondary', 'coefficient': 0.9815238995422373, 'lower': 0.9802831270349176, 'upper': 0.982764672049557}
    % {'feature': 'school_type_vec_State-funded primary', 'coefficient': 0.9414428929574084, 'lower': 0.9402541524476719, 'upper': 0.9426316334671448}
    % {'feature': 'enrolments', 'coefficient': 1.0000185636282746, 'lower': 1.000018347212287, 'upper': 1.0000187800442621}
    % {'feature': 'sess_unauthorised_percent', 'coefficient': 0.9865300767465852, 'lower': 0.9864508352203877, 'upper': 0.9866093182727828}
    % {'feature': 'sess_overall_percent_pa_10_exact', 'coefficient': 1.0108745301099418, 'lower': 1.0108538668092961, 'upper': 1.0108951934105874}
    % {'feature': 'sess_unauthorised_percent_pa_10_exact', 'coefficient': 1.0062488225278534, 'lower': 1.0062211970407513, 'upper': 1.0062764480149555}
    % {'feature': 'enrolments_pa_10_exact_percent', 'coefficient': 1.0306596159912702, 'lower': 1.030648784911579, 'upper': 1.0306704470709613}
    % {'feature': 'region_name_school_type_interaction_region_name_vec_Yorkshire and the Humber:school_type_vec_State-funded secondary', 'coefficient': 0.9807420889768835, 'lower': 0.9790617758092173, 'upper': 0.9824224021445498}
    % {'feature': 'region_name_school_type_interaction_region_name_vec_Yorkshire and the Humber:school_type_vec_State-funded primary', 'coefficient': 0.9852478464224804, 'lower': 0.9835568284798909, 'upper': 0.98693886436507}
    % {'feature': 'region_name_school_type_interaction_region_name_vec_West Midlands:school_type_vec_State-funded secondary', 'coefficient': 0.9750722219537781, 'lower': 0.9735756355023338, 'upper': 0.9765688084052223}
    % {'feature': 'region_name_school_type_interaction_region_name_vec_West Midlands:school_type_vec_State-funded primary', 'coefficient': 0.9848571965591602, 'lower': 0.9833431401941385, 'upper': 0.9863712529241818}
    % {'feature': 'region_name_school_type_interaction_region_name_vec_South West:school_type_vec_State-funded secondary', 'coefficient': 1.0336781329267763, 'lower': 1.0318933097126985, 'upper': 1.0354629561408542}
    % {'feature': 'region_name_school_type_interaction_region_name_vec_South West:school_type_vec_State-funded primary', 'coefficient': 1.051217008801224, 'lower': 1.0493985783170745, 'upper': 1.0530354392853736}
    % {'feature': 'region_name_school_type_interaction_region_name_vec_South East:school_type_vec_State-funded secondary', 'coefficient': 0.9972682182167923, 'lower': 0.9957874455228126, 'upper': 0.9987489909107721}
    % {'feature': 'region_name_school_type_interaction_region_name_vec_South East:school_type_vec_State-funded primary', 'coefficient': 0.9932052137658097, 'lower': 0.9917283635378756, 'upper': 0.9946820639937438}
    % {'feature': 'region_name_school_type_interaction_region_name_vec_Outer London:school_type_vec_State-funded secondary', 'coefficient': 1.0168461017528507, 'lower': 1.0151389669321105, 'upper': 1.018553236573591}
    % {'feature': 'region_name_school_type_interaction_region_name_vec_Outer London:school_type_vec_State-funded primary', 'coefficient': 1.0280457552594142, 'lower': 1.0263190617438807, 'upper': 1.0297724487749478}
    % {'feature': 'region_name_school_type_interaction_region_name_vec_North West:school_type_vec_State-funded secondary', 'coefficient': 1.0130761729470235, 'lower': 1.011534723981762, 'upper': 1.0146176219122849}
    % {'feature': 'region_name_school_type_interaction_region_name_vec_North West:school_type_vec_State-funded primary', 'coefficient': 1.0129505289826333, 'lower': 1.01140625115817, 'upper': 1.0144948068070965}
    % {'feature': 'region_name_school_type_interaction_region_name_vec_North East:school_type_vec_State-funded secondary', 'coefficient': 0.9932256921569959, 'lower': 0.9913996062939359, 'upper': 0.9950517780200558}
    % {'feature': 'region_name_school_type_interaction_region_name_vec_North East:school_type_vec_State-funded primary', 'coefficient': 1.0139044580125303, 'lower': 1.0120334864692226, 'upper': 1.0157754295558379}
    % {'feature': 'region_name_school_type_interaction_region_name_vec_Inner London:school_type_vec_State-funded secondary', 'coefficient': 1.0237481925188423, 'lower': 1.0217541666208332, 'upper': 1.0257422184168514}
    % {'feature': 'region_name_school_type_interaction_region_name_vec_Inner London:school_type_vec_State-funded primary', 'coefficient': 1.0424177657028986, 'lower': 1.0403919543302649, 'upper': 1.0444435770755323}
    % {'feature': 'region_name_school_type_interaction_region_name_vec_East of England:school_type_vec_State-funded secondary', 'coefficient': 0.9835771443122107, 'lower': 0.9819554171182295, 'upper': 0.9851988715061918}
    % {'feature': 'region_name_school_type_interaction_region_name_vec_East of England:school_type_vec_State-funded primary', 'coefficient': 0.9904061776262151, 'lower': 0.9887699080518163, 'upper': 0.9920424472006139}


\begin{longtable}{|p{0.5\textwidth}|p{0.2\textwidth}|p{0.2\textwidth}|}
    
        \hline
        \textbf{Feature} & \textbf{Coefficient} & \textbf{P-value} \\  
        \hline
        \endfirsthead

        time\_period\_201819 & 0.95 & 0.00 \\
        time\_period\_201718 & 0.97 & 0.00 \\
        time\_period\_201617 & 0.95 & 0.00 \\
        time\_period\_201516 & 0.95 & 0.00 \\
        time\_period\_201415 & 0.95 & 0.00 \\
        time\_period\_201314 & 0.93 & 0.00 \\
        time\_period\_201213 & 1.00 & 0.00 \\
        time\_period\_201112 & 0.97 & 0.00 \\
        time\_period\_201011 & 1.01 & 0.00 \\
        time\_period\_200910 & 1.01 & 0.00 \\
        time\_period\_200809 & 1.01 & 0.00 \\
        \hline

        region\_name\_Yorkshire and the Humber & 1.02 & 0.00 \\
        region\_name\_West Midlands & 1.02 & 0.00 \\
        region\_name\_South West & 0.98 & 0.00 \\
        region\_name\_South East & 1.01 & 0.00 \\
        region\_name\_Outer London & 0.98 & 0.00 \\
        region\_name\_North West & 0.98 & 0.00 \\
        region\_name\_North East & 1.01 & 0.00 \\
        region\_name\_Inner London & 0.96 & 0.00 \\
        region\_name\_East of England & 1.03 & 0.00 \\
        \hline

        school\_type\_State-funded secondary & 0.98 & 0.00 \\
        school\_type\_State-funded primary & 0.94 & 0.00 \\
        \hline

        enrolments & 1.00 & 0.00 \\
        unauthorised\_percent & 0.99 & 0.00 \\
        overall\_percent\_persistent\_absentees & 1.01 & 0.00 \\
        unauthorised\_percent\_persistent\_absentees & 1.01 & 0.00 \\
        enrolments\_persistent\_absentees\_percent & 1.03 & 0.00 \\
        \hline

        Yorkshire and the Humber:State-funded secondary & 0.98 & 0.00 \\
        West Midlands:State-funded secondary & 0.98 & 0.00 \\
        South West:State-funded secondary & 1.03 & 0.00 \\
        South East:State-funded secondary & 1.00 & 0.00 \\
        Outer London:State-funded secondary & 1.02 & 0.00 \\
        North West:State-funded secondary & 1.01 & 0.00 \\
        North East:State-funded secondary & 0.99 & 0.00 \\
        Inner London:State-funded secondary & 1.02 & 0.00 \\
        East of England:State-funded secondary & 0.98 & 0.00 \\
        \hline
        
        Yorkshire and the Humber:State-funded primary & 0.99 & 0.00 \\ 
        West Midlands:State-funded primary & 0.98 & 0.00 \\
        South West:State-funded primary & 1.05 & 0.00 \\
        South East:State-funded primary & 0.99 & 0.00 \\
        Outer London:State-funded primary & 1.03 & 0.00 \\
        North West:State-funded primary & 1.01 & 0.00 \\
        North East:State-funded primary & 1.01 & 0.00 \\
        Inner London:State-funded primary & 1.04 & 0.00 \\
        East of England:State-funded primary & 0.99 & 0.00
        \hline

        \caption{Table of all coefficients and p-values for the model of absence rates. The reference level for region name is East Midlands, for school type is special school, and for year is 2007/08. Coefficients are the estimated effect size of the variable on the absence rate, e.g state funded primary schools in Yorkshire and the Humber have 1\% lower absence rates than special schools in East Midlands. All variables are significant predictors of absence rates ($p < 0.001$).}
    \label{tab:model_output}
\end{longtable}

\begin{figure}[htbp]
    \centering
    \includegraphics[width=\textwidth]{img/absence\_rate\_by\_region.png}
    \caption{Bar chart of overall absence rates by region, with a line representing the mean absence rate and a shaded area representing the 95\% confidence interval of the mean. Values below or above the shaded area are significantly different from the mean region's absence rate, e.g Inner London has significantly lower absence rates than the mean region.}
    \label{fig:absence_rate_by_region}
\end{figure}

\begin{figure}[htbp]
    \centering
    \includegraphics[width=\textwidth]{img/absence\_rate\_by\_school\_type.png}
    \caption{Bar chart of overall absence rates by school type, with a line representing the mean absence rate and a shaded area representing the 95\% confidence interval of the mean. Values below or above the shaded area are significantly different from the mean absence rate, e.g special schools have significantly higher absence rates than the mean school type. However, this chart is a misleading representation (see Discussion).}
    \label{fig:absence_rate_by_school_type}
\end{figure}

\begin{figure}[htbp]
    \centering
    \includegraphics[width=\textwidth]{img/absence\_rate\_over\_time\_by\_region.png}
    \caption{Line chart of overall absence rates over time by region, with a line representing the mean absence rate and a shaded area representing the 95\% confidence interval of the mean. Values below or above the shaded area are significantly different from the mean region's absence rate for that year, e.g Yorkshire and the Humber has significantly higher absence rates than the mean region in 2006/07. A mean absence rate for one year is significantly different from another year if the new mean is outside the shaded area of the old mean, e.g the mean absence rate in 2006/07 is significantly higher than the mean absence rate in 2007/08.}
    \label{fig:absence_rate_over_time_by_region}
\end{figure}

\begin{figure}[htbp]
    \centering
    \includegraphics[width=\textwidth]{img/proportion\_school\_types\_by\_region.png}
    \caption{Bar chart of the proportion of each school type by region, with a line representing the mean proportion of each school type and a shaded area representing the 95\% confidence interval of the mean. Values below or above the shaded area are significantly different from the mean proportion of that school type, e.g Inner London has significantly more special schools than the mean region.}
    \label{fig:proportion_school_types_by_region}
\end{figure}

\end{document}
